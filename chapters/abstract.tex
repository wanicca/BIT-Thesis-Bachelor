%%==================================================
%% abstract.tex for BIT Master Thesis
%% modified by yang yating
%% version: 0.1
%% last update: Dec 25th, 2016
%%==================================================

\begin{abstract}
DRM即数字调幅广播,是数字化广播的候选方案之一。它提供新业务,同时仅改变发射系统的信源、编码和调制模块。通常数字音频质量由数据率决定,然而,高数据率需要高传输带宽。因此,需要对数字音频信号进行压缩,即信源编码。信源编码是DRM的关键技术之一,编码质量直接决定了声音信号的质量和系统传输所需要的频带宽度。

本文阐述了MPEG-4音频编码中CELP编解码的原理、算法和验证。作者阅读了大量MPEG-4中CELP的语音编码标准及有关文献。并在VC的编程环境下,用C语言编程实现MPEG-4中码激励线性预测的编解码。

在论文安排上,首先,作者简单介绍了论文背景和所做工作;其次,较深入地阐述了CELP编码原理;然后,详细讲述了MPEG-4音频编码中CELP的实现原理;接着,讲述了CELP编解码在PC机上的详细实现过程;最后,对实验结果进行了分析和主观评估。

\keywords{语音编码;CELP;MPEG-4}
\end{abstract}

\begin{englishabstract}
DRM,which means Digital AM Broadcasting,is one of the digital broadcasting schemes.DRM system supports new bussiness through changing only message resource,coding and modulation modules.Although digital audio quality is determined by data rate,high data rate requires high transfer bandwidth.So,we need compress digital audio signal,that is source coding.Source coding is one of the key technologies in DRM,the quality of source coding determines the quality of the speech and the required system bandwidth.

This paper puts emphasis on theory, algorithm and verifications of CELP codec in speech coding of MPEG-4.The author read a great amount of literature and standard papers and then programmed and made the realization of CELP codec.

In this paper,the author firstly introduces the background and his work.Then,he details the theory of CELP coding.Thirdly,the author discusses realization theory of CELP in MPEG-4 speech coding.Next,he presents the realization process of CELP coding at length.Finally,the author analyses and subjectively evaluates the result.
   
\englishkeywords{speech coding;CELP;MPEG-4}

\end{englishabstract}
